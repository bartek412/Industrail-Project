\documentclass[11pt,a4paper]{article}

\usepackage[english]{babel}
\usepackage{polski}
\usepackage[utf8]{inputenc}
\usepackage[cm]{fullpage}
\title{Industrial Project Report}
\author{Maciej Stępień, Adam Sztamborski, Bartłomiej Gryglak}
\date{}

\begin{document}
\maketitle
\pagebreak

\tableofcontents
\pagebreak

\begin{abstract}
Dokument ten prezentuje kilka zasad składu tekstu
w~systemie \LaTeX.
\end{abstract}
% pierwsza sekcja
\section{Tekst}\label{sec:tekst}
\LaTeX\ ułatwia autorowi tekstu zarządzanie
numerowaniem sekcji, wypunktowaniami oraz odwołaniami
do tabel, rysunków i~innych elementów. W~łatwy sposób
możemy się odwołać do wzoru \ref{eqn:wzor1}.
\pagebreak

% druga sekcja
\section{Matematyka}\label{sec:matematyka}
Poniższy wzór prezentuje możliwości \LaTeX\ w~zakresie
składu formuł matematycznych. Wzory są numerowane
automatycznie, podobnie jak inne elementy o~których
mowa w~sekcji~\ref{sec:tekst}.
\begin{equation}
    E = mc^2,
    \label{eqn:wzor1}
\end{equation}
gdzie
\begin{equation}
    m = \frac{m_0}{\sqrt{1-\frac{v^2}{c^2}}}.
\end{equation}


\end{document}